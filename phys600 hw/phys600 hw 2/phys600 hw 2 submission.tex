%%%%%%%%%%%%%%%%%%%%%%%%%%%%%%%%%%%%%%%%%%%%%%%
%%%This is a science homework template. Modify the preamble to suit your needs. 
%The junk text is   there for you to immediately see how the headers/footers look at first 
%typesetting.

\documentclass[12pt]{article}

%AMS-TeX packages
\usepackage{amssymb,amsmath,amsthm} 
\usepackage{commath}
\usepackage[margin=1in]{geometry}
\usepackage{graphicx,ctable,booktabs}
\usepackage[retainorgcmds]{IEEEtrantools}
\usepackage{cancel}
\usepackage{wrapfig}
\usepackage{braket}
\usepackage{enumitem}
\usepackage{pdfpages}

\usepackage{graphicx}
\usepackage{subfig}

%Redefining sections as problems

\makeatletter
\newenvironment{problem}{\@startsection
	{section}
	{1}
	{-.2em}
	{-3.5ex plus -1ex minus -.2ex}
	{2.3ex plus .2ex}
	{\pagebreak[3]%forces pagebreak when space is small; use \eject for better results
		\large\bf\noindent{Problem }
	}
}
{%\vspace{1ex}\begin{center} \rule{0.3\linewidth}{.3pt}\end{center}}
	\begin{center}\large\bf \ldots\ldots\ldots\end{center}}
\makeatother

%Fancy-header package to modify header/page numbering 

\usepackage{fancyhdr}
\pagestyle{fancy}
\lhead{Problem \thesection}
\chead{} 
\rhead{\thepage} 
\lfoot{\small\scshape PHYS 600} 
\cfoot{} 
\rfoot{\footnotesize HW 2} 
\renewcommand{\headrulewidth}{.3pt} 
\renewcommand{\footrulewidth}{.3pt}
\setlength\voffset{-0.25in}
\setlength\textheight{648pt}
\allowdisplaybreaks

\newcommand{\partder}[3]{\ensuremath{\left(\frac{\partial {#1}}{\partial {#2}}\right)_{#3}}}

\newcommand{\braks}[1]{\ensuremath{\left\langle{#1} \right\rangle} }

\setlength{\parindent}{0pt} % No indent by default

%%%%%%%%%%%%%%%%%%%%%%%%%%%%%%%%%%%%%%%%%%%%%%%

%
%Contents of problem set
%    
\begin{document}
	
	\title{PHYS 600: Homework 2}
	\author{Yarone Tokayer}
	\date{September 22, 2023}
	
	\maketitle
	
	\thispagestyle{empty}

	\begin{problem}{Friedmann Equation II}
		We wish to derive \begin{equation} \label{eq:f2}
			\frac{\ddot{a}}{a} = -\frac{4\pi G}{3}(\rho + 3P)
		\end{equation} from the equations \begin{align}
			\left(\frac{\dot{a}}{a}\right)^2 &= \frac{8\pi G}{3}\rho - \frac{\kappa}{R_0^2a^2} \label{eq:f1}
			\\
			0 &= \dot{\rho} + 3H(\rho+P) \label{eq:cont}
		\end{align}
		
		Begin by taking the time derivative on both sides of Eq.~\ref{eq:f1}: \begin{align*}
			2 H \frac{a\ddot{a} - \dot{a}^2}{a^2} &= \frac{8\pi G}{3}\dot{\rho} - 2 H \frac{\kappa}{R_0^2a^2} & \textit{$\left(H = \frac{\dot{a}}{a}\right)$}
			\\
			2H \frac{\ddot{a}}{a} - 2H^3 &= -3H \frac{8\pi G}{3}(\rho + P) - 2 H \frac{\kappa}{R_0^2a^2}  & \textit{(Eq.~\ref{eq:cont})}
			\\
			\frac{\ddot{a}}{a} - H^2 &= -\frac{8\pi G}{3}\rho - \frac{4\pi G}{3}\rho -3 \frac{4\pi G}{3} P -  \frac{\kappa}{R_0^2a^2}
			\\
			\frac{\ddot{a}}{a} - \cancel{H^2} &= -\cancel{H^2} - \frac{4\pi G}{3}\rho -3 \frac{4\pi G}{3} P & \textit{(Eq.~\ref{eq:f1})}
			\\
			\frac{\ddot{a}}{a} &= -\frac{4\pi G}{3}(\rho + 3P)
		\end{align*}
	
	\end{problem}
	
	\begin{problem}{Cosmological Dimming}
		We saw in class that the angular diameter distance goes as $(1+z)^{-1}$, which implies that angular size goes as $(1+z)$. Additionally, we note that the bolometric luminosity $L$ scales as $(1+z)^{-2}$, where the two factors of $(1+z)$ are due to cosmological redshift and hubble drag, respectively.\\
		
		By definition, the bolometric surface brightness of an object, $I_\mathrm{e}$ is given by \begin{equation*}
			I_\mathrm{e} = \frac{L}{4\pi r^2},
		\end{equation*} where $L$ is the intrinsic bolometric luminosity and $r$is the radius of the object.  Using the scalings above, we then get for the observed surface brightness: \begin{align*}
			I_\mathrm{o} &= \frac{L (1+z)^{-2} }{4\pi \left(r(1 + z)\right)^2}
			\\
			&= I_\mathrm{e} (1+z)^{-4}
		\end{align*}
		
	\end{problem}
	
	\begin{problem}{Magnitudes and K-corrections}
		\begin{itemize}
			\item \begin{align*}
				m &= -2.5 \log \left[\frac{f}{f_0}\right]
				\\
				&= -2.5 \log \left[\frac{f}{f(10\text{ pc})}\frac{f(10\text{ pc})}{f_0}\right]
				\\
				&= -2.5 \log \left[\frac{f(10\text{ pc})}{f_0}\right] -2.5 \log \left[\frac{f}{f(10\text{ pc})}\right] 
				\\
				&= M -2.5 \log \left[\frac{10\text{ pc}}{D_\mathrm{L}(z)}\right]^2
				\\
				&= M +5 \log \left[\frac{D_\mathrm{L}(z)}{10\text{ pc}}\right]
				\\
				&= M + \mathrm{DM}(z)
			\end{align*}
			
			\item In general, flux $S$ is related to bolometric luminosity $L$ by \begin{equation*}
				S = \frac{L}{4\pi D_\mathrm{L}^2}
			\end{equation*} For a flux in the frequency interval $(\nu,\nu + \dif\nu)$, we must look not at the energy band being observed, but the emitted energy band, which is given by $\nu(1 + z)$, hence the intrinsic luminosity band we are interested in is $L_{\nu(1+z)}$, and we must multiply by the ratio $L_{\nu(1+z)}/L_\nu$.  Finally, we add a factor of $(1+z)$ to account for the fact that the photons have redshifted, so we need to convert back to the restframe.  So we get:  \begin{equation*}
				S_\nu = (1+z) \frac{L_{\nu(1+z)}}{L_\nu} \frac{L_\nu}{4\pi D_\mathrm{L}^2}
			\end{equation*}
		\end{itemize}
		
	\end{problem}
	
	\begin{problem}{A Static Universe}
		\begin{itemize}
			\item Begin by differentiating Friedmann I with respect to time: \begin{align*}
				\frac{\dif}{\dif t} \left[\left(\frac{\dot{a}}{a}\right)^2\right] &= \frac{\dif}{\dif t}  \left[\frac{8\pi G}{3}\left(\rho_\mathrm{M} + \rho_\Lambda\right) - \frac{k}{a^2}\right]
				\\
				2 \frac{\ddot{a}}{a} \frac{\dot{a}}{a} &= \frac{8\pi G}{3}\left(\dot{\rho_\mathrm{M}} + \dot{\rho_\Lambda}\right) - \left(-2k\frac{\dot{a}}{a^3}\right)
				\\
				2 \frac{\ddot{a}}{a} \frac{\dot{a}}{a} &= \frac{8\pi G}{3} \left[-3\frac{\dot{a}}{a}\left(\rho_\mathrm{M} + \cancelto{0}{P_\mathrm{M}} + \rho_\Lambda - \rho_\Lambda \right) \right]+ 2k\frac{\dot{a}}{a^3}
				\\
				\implies \frac{\ddot{a}}{a}&= -4\pi G\rho_\mathrm{M} + \frac{k}{a^2}
			\end{align*} We used the continuity equation with $P_\mathrm{M}=0$ and $P_\Lambda = -\rho_\Lambda$ in the third line.
			
			\item We wish the solve the following system of equations for $\Lambda$ and $k$: \begin{align*}
				0 &= \frac{8\pi G}{3}\rho_\mathrm{M} + \frac{\Lambda}{3} - \frac{k}{a^2}
				\\
				0 &= -4\pi G\rho_\mathrm{M} + \frac{k}{a^2}
			\end{align*} Add them: \begin{align*}
				0 &= -\frac{4\pi G}{3}\rho_\mathrm{M} + \frac{\Lambda}{3}
				\\
				\implies \Lambda &= 4\pi G \rho_\mathrm{M}
				\\
				\implies k &= 4\pi Ga^2\rho_\mathrm{M}
				\\
				&= a^2\Lambda
			\end{align*} Since $k>0$, this is a closed universe.
			
			\item Substituting into the second Friedmann equation, we get \begin{align*}
				\frac{\dif^2}{\dif t^2} \left[\delta a(t)\right] &= -4\pi G \rho_\mathrm{M}(t) a(t) + \frac{k}{a(t) }
				\\
				&= -\Lambda \left[ 1 - 3\delta a(t) \right] \left[ 1 + \delta a(t) \right] + \Lambda  \left[ 1 + \delta a(t) \right] 
				\\
				&= \Lambda \left[ 2\delta a(t) - 1 \right] + \Lambda  \left[ 1 + \delta a(t) \right] 
				\\
				\ddot{\delta a(t)} &= 3 \Lambda \delta a(t)
				\\
				\implies \delta a(t) &= A e^{\sqrt{3\Lambda} t} + B e^{-\sqrt{3\Lambda} t}
			\end{align*}  Applying the initial conditions $\delta a(t_0) = \delta a_0$ and $\dot{\delta a}(t_0) = 0$, we get $A = \frac{1}{2}\delta a_0 e^{-\sqrt{3\Lambda} t_0}$ and $B = \frac{1}{2}\delta a_0 e^{\sqrt{3\Lambda} t_0}$.  Thus \begin{equation*}
				\delta a(t) =  \frac{\delta a_0}{2}  \left[e^{\sqrt{3\Lambda} (t-t_0)} + e^{-\sqrt{3\Lambda} (t-t_0)}\right]
			\end{equation*} Since $\Lambda>0$ $(\impliedby \rho_\mathrm{M}>0)$, the perturbation grows exponentially and the solution is unstable.
		\end{itemize}
		
	\end{problem}
	
	\begin{problem}{Redshift Drift}
		Start with the definition of redshift as advised: \begin{align*}
			\frac{\dif}{\dif t_0}\left[ 1 + z \right] &= \frac{\dif}{\dif t_0}\left[ \frac{a(t_0)}{a(t_1)} \right]
			\\
			\frac{\dif z}{\dif t_0} &= \frac{1}{a(t_1)} \frac{\dif a(t_0)}{\dif t_0} - \frac{a(t_0)}{a(t_1)^2} \frac{\dif a(t_1)}{\dif t_1}\frac{\dif t_1}{\dif t_0}
			\\
			&= \frac{a_0}{a(t_1)} \frac{\dot{a_0}}{a_0} - \frac{a(t_0)}{a(t_1)} \frac{\dot{a}(t_1)}{a(t_1)}\frac{\dif t_1}{\dif t_0}
			\\
			&= (1+z) H_0 - (1+z)H(t_1)(1+z)^{-1}
			\\
			&= (1+z) H_0 - H(t_1)
		\end{align*} We have used $\dif t_1/\dif t_0 =(1+z)^{-1}$, which is just the time dilation formula, and be derived using dimensional analysis: \begin{align*}
			1 + z &= \frac{\lambda_0}{\lambda_1}
			\\
			\implies 1 + z &= \frac{\nu_1}{\nu_0}
		\end{align*} but by definition of frequency (and by virtue of its units) $\nu\propto \frac{1}{\dif t}$, so we have \begin{align*}
			1 + z &= \frac{\dif t_0}{\dif t_1}.
		\end{align*}
		
		Assuming a matter-dominated flat universe, we have $H(z) = H_0(1+z)^{3/2}$.  So at $z=1$, \begin{align*}
			\left.\frac{\dif z}{\dif t_0}\right\vert_{z=1} &= \left(2 - 2^{3/2}\right)H_0
			\\
			&\approx -0.828 H_0
			\\
			&= -82.8 h \frac{\text{km}}{\text{s Mpc}}
			\\
			&\approx -8.462 \time 10^{-11} h \frac{1}{\text{yr}}
		\end{align*} According to the paper cited in the problem set, Keck spectra in 2003 could measure $z$ to a precision of $10^{-5}$.  So we would need to wait $10^6$ years before we would be able to detect redshift drift for sources at $z=1$ using that technology.
		
	\end{problem}

% Appendix
% \includepdf[pages=-]{/Users/yaronetokayer/Yale Drive/Classes/ASTR 610/astr610 ps/astr610 ps 4/astr610 ps 4.pdf}

\end{document}