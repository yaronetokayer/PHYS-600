%%%%%%%%%%%%%%%%%%%%%%%%%%%%%%%%%%%%%%%%%%%%%%%
%%%This is a science homework template. Modify the preamble to suit your needs. 
%The junk text is   there for you to immediately see how the headers/footers look at first 
%typesetting.

\documentclass[12pt]{article}

%AMS-TeX packages
\usepackage{amssymb,amsmath,amsthm} 
\usepackage{commath}
\usepackage[margin=1in]{geometry}
\usepackage{graphicx,ctable,booktabs}
\usepackage[retainorgcmds]{IEEEtrantools}
\usepackage{cancel}
\usepackage{wrapfig}
\usepackage{braket}
\usepackage{enumitem}
\usepackage{pdfpages}

\usepackage{graphicx}
\usepackage{subfig}

%Redefining sections as problems

\makeatletter
\newenvironment{problem}{\@startsection
	{section}
	{1}
	{-.2em}
	{-3.5ex plus -1ex minus -.2ex}
	{2.3ex plus .2ex}
	{\pagebreak[3]%forces pagebreak when space is small; use \eject for better results
		\large\bf\noindent{Problem }
	}
}
{%\vspace{1ex}\begin{center} \rule{0.3\linewidth}{.3pt}\end{center}}
	\begin{center}\large\bf \ldots\ldots\ldots\end{center}}
\makeatother

%Fancy-header package to modify header/page numbering 

\usepackage{fancyhdr}
\pagestyle{fancy}
\lhead{Problem \thesection}
\chead{} 
\rhead{\thepage} 
\lfoot{\small\scshape PHYS 600} 
\cfoot{} 
\rfoot{\footnotesize HW 2} 
\renewcommand{\headrulewidth}{.3pt} 
\renewcommand{\footrulewidth}{.3pt}
\setlength\voffset{-0.25in}
\setlength\textheight{648pt}
\allowdisplaybreaks

\newcommand{\partder}[3]{\ensuremath{\left(\frac{\partial {#1}}{\partial {#2}}\right)_{#3}}}

\newcommand{\braks}[1]{\ensuremath{\left\langle{#1} \right\rangle} }

\setlength{\parindent}{0pt} % No indent by default

%%%%%%%%%%%%%%%%%%%%%%%%%%%%%%%%%%%%%%%%%%%%%%%

%
%Contents of problem set
%    
\begin{document}
	
	\title{PHYS 600: Homework 2}
	\author{Yarone Tokayer}
	\date{September 22, 2023}
	
	\maketitle
	
	\thispagestyle{empty}

	\begin{problem}{Friedmann Equation II}
		We wish to derive \begin{equation} \label{eq:f2}
			\frac{\ddot{a}}{a} = -\frac{4\pi G}{3}(\rho + 3P)
		\end{equation} from the equations \begin{align}
			\left(\frac{\dot{a}}{a}\right)^2 &= \frac{8\pi G}{3}\rho - \frac{\kappa}{R_0^2a^2} \label{eq:f1}
			\\
			0 &= \dot{\rho} + 3H(\rho+P) \label{eq:cont}
		\end{align}
		
		Begin by taking the time derivative on both sides of Eq.~\ref{eq:f1}: \begin{align*}
			2 H \frac{a\ddot{a} - \dot{a}^2}{a^2} &= \frac{8\pi G}{3}\dot{\rho} - 2 H \frac{\kappa}{R_0^2a^2} & \textit{$\left(H = \frac{\dot{a}}{a}\right)$}
			\\
			2H \frac{\ddot{a}}{a} - 2H^3 &= -3H \frac{8\pi G}{3}(\rho + P) - 2 H \frac{\kappa}{R_0^2a^2}  & \textit{(Eq.~\ref{eq:cont})}
			\\
			\frac{\ddot{a}}{a} - H^2 &= -\frac{8\pi G}{3}\rho - \frac{4\pi G}{3}\rho -3 \frac{4\pi G}{3} P -  \frac{\kappa}{R_0^2a^2}
			\\
			\frac{\ddot{a}}{a} - \cancel{H^2} &= -\cancel{H^2} - \frac{4\pi G}{3}\rho -3 \frac{4\pi G}{3} P & \textit{(Eq.~\ref{eq:f1})}
			\\
			\frac{\ddot{a}}{a} &= -\frac{4\pi G}{3}(\rho + 3P)
		\end{align*}
	
	\end{problem}
	
	\begin{problem}{Cosmological Dimming}
		We saw in class that the angular diameter distance goes as $(1+z)^{-1}$, which implies that angular size goes as $(1+z)$. Additionally, we note that the bolometric luminosity $L$ scales as $(1+z)^{-2}$, where the two factors of $(1+z)$ are due to cosmological redshift and hubble drag, respectively.\\
		
		By definition, the bolometric surface brightness of an object, $I_\mathrm{e}$ is given by \begin{equation*}
			I_\mathrm{e} = \frac{L}{4\pi r^2},
		\end{equation*} where $L$ is the intrinsic bolometric luminosity and $r$is the radius of the object.  Using the scalings above, we then get for the observed surface brightness: \begin{align*}
			I_\mathrm{o} &= \frac{L (1+z)^{-2} }{4\pi \left(r(1 + z)\right)^2}
			\\
			&= I_\mathrm{e} (1+z)^{-4}
		\end{align*}
		
	\end{problem}



% Appendix
% \includepdf[pages=-]{/Users/yaronetokayer/Yale Drive/Classes/ASTR 610/astr610 ps/astr610 ps 4/astr610 ps 4.pdf}

\end{document}